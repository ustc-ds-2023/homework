\documentclass[14pt]{extarticle}
\usepackage[margin=0.1in,top=0.05in,bottom=0.05in]{geometry}
\usepackage{graphicx}
\usepackage{minted}
\usemintedstyle{friendly}
\setminted{breaklines,fontsize={\fontsize{12.5pt}{13pt}\selectfont},frame=single,framesep=10pt}
\usepackage{fontspec}
\setmainfont{Source Sans Pro}
\setmonofont{Maple Mono SC NF}
\usepackage{xeCJK}
\setCJKmainfont{Source Han Sans SC}
\usepackage{float}
\usepackage{titlesec}
\usepackage{amsmath}
\titlespacing\section{0pt}{30pt plus 4pt minus 2pt}{5pt plus 2pt minus 2pt}
\usepackage{hyperref}
\makeatletter
\renewcommand\@seccntformat[1]{}
\makeatother

\linespread{1.5}

\begin{document}

\section{7.11}
\begin{figure}[H]
        \centering
        \includegraphics[width=1.0\linewidth]{/home/yu/.latex-images/2023-12-17/1702817547109.png}
\end{figure}

\begin{figure}[H]
        \centering
        \includegraphics[width=1.0\linewidth]{/home/yu/.latex-images/2023-12-17/1702817590698.png}
\end{figure}

\begin{minted}{text}
节点 a 的最短距离为 0,尝试以该节点为源点优化路径
        从节点 a 到节点 b 可使节点 b 距离更近(inf -> 15),更新节点 b 距离
        从节点 a 到节点 c 可使节点 c 距离更近(inf -> 2),更新节点 c 距离
        从节点 a 到节点 d 可使节点 d 距离更近(inf -> 12),更新节点 d 距离
节点 c 的最短距离为 2,尝试以该节点为源点优化路径
        从节点 c 到节点 e 可使节点 e 距离更近(inf -> 10),更新节点 e 距离
        从节点 c 到节点 f 可使节点 f 距离更近(inf -> 6),更新节点 f 距离
节点 f 的最短距离为 6,尝试以该节点为源点优化路径
        从节点 f 到节点 d 可使节点 d 距离更近(12 -> 11),更新节点 d 距离
        从节点 f 到节点 g 可使节点 g 距离更近(inf -> 16),更新节点 g 距离
节点 e 的最短距离为 10,尝试以该节点为源点优化路径
节点 d 的最短距离为 11,尝试以该节点为源点优化路径
        从节点 d 到节点 g 可使节点 g 距离更近(16 -> 14),更新节点 g 距离
节点 g 的最短距离为 14,尝试以该节点为源点优化路径
节点 b 的最短距离为 15,尝试以该节点为源点优化路径
\end{minted}

\section{9.26}
\begin{figure}[H]
        \centering
        \includegraphics[width=1.0\linewidth]{/home/yu/.latex-images/2023-12-17/1702817648934.png}
\end{figure}

\inputminted{cpp}{9.26.cpp}

\end{document}

